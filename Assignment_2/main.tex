\documentclass{article}
\usepackage[utf8]{inputenc}

\title{CS6910 Fundamentals of Deep Learning \\ Assignment2}
\author{Lakshya J(EE19B035)\\ Neham Jain(EE19B084)\\ Nisharg Manvar(EE19B094)}
\date{\today}

\begin{document}

\maketitle
\section{Task 2}
\subsection{About}
\begin{itemize}
    \item In this part we design a Stacked encoder of 3 layers using pre-trained Auto Associative Neural Networks (AANNs) of depth 5. Each of the three AANNs are trained with the output of the bottleneck layer of the previous AANN (except the first one, which is trained using the dataset provided).
    \item On observation of the datasets given, the row of each of the classes sums to a particular value. Hence we decided to normalize each of the rows with the sum of the elements in each row.
    \item On running the simulations for multiple learning rates and different batch sizes we find the optimal hyperparameters.
\end{itemize}

\subsection{Model}
\begin{itemize}
    \item Since the input size is 48, we use autoencoders with input length d=48, l1 = 32, l2 = 16, l3 = 8. These are the layer sizes we obtained after performing PCA from the previous dataset. Finally we add a linear layer of size 5, which is the number of classes, to approximate the class which the input data belongs.
\end{itemize}

\subsection{Observations}
\begin{itemize}
    \item 
\end{itemize}

\section*{Task 3: Using Deep CNN Features of VGGNet and GoogLeNet for Image Classification}

\subsection*{Dataset Details}

The given data-set has a set of 10947 images in total and 5 classes.

We use this dataset to train our model to classify the input images into 
one of the 5 classes.

\subsection*{Implementation Details}


\subsection*{Results}



\subsection*{Experimental Analysis}

Usage of data augmentation 

Taking 

\subsection*{Observations}


\section*{Task 4: Using custom CNN model with 2 convolutional layers for Image Classification}

\subsection*{Implementation Details}


\subsection*{Results}

\subsection*{Experimental Analysis}
Usage of data augmentation 
Taking
\subsection*{Observations}



\end{document}